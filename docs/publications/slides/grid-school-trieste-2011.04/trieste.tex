% Created 2011-04-21 gio 12:11
\documentclass[presentation]{beamer}
\usepackage[utf8]{inputenc}
\usepackage[T1]{fontenc}
\usepackage{fixltx2e}
\usepackage{graphicx}
\usepackage{longtable}
\usepackage{float}
\usepackage{wrapfig}
\usepackage{soul}
\usepackage{textcomp}
\usepackage{marvosym}
\usepackage{wasysym}
\usepackage{latexsym}
\usepackage{amssymb}
\usepackage{hyperref}
\tolerance=1000
\providecommand{\alert}[1]{\textbf{#1}}
\usepackage{pslatex}\usetheme{CambridgeUS}\usecolortheme{dolphin}\setbeamertemplate{navigation symbols}{}\setlength{\parskip}{1em}
\begin{document}



\title{GC3Pie}
\author{Riccardo Murri, GC3, University of Zurich}
\date{2011-04-21}
\maketitle





\begin{frame}
\frametitle{A typical high-throughput use case?}
\label{sec-1}


  Run application \emph{A} on a range of different inputs.
  Each input is a different file (or a set of files).

  Then collect output files and post-process them, e.g., gather some
  statistics.

  Typically implemented by a set of \texttt{sh} / \texttt{perl} scripts to drive
  execution on a local cluster.
\end{frame}
\begin{frame}
\frametitle{Potential issues}
\label{sec-2}


\begin{enumerate}
\item \textbf{Portability:} Cannot run on a different cluster without rewriting
   all the scripts.
\item \textbf{Code reuse:} Scripts are often very tied to a certain purpose, so
   they are difficult to reuse.
\item \textbf{Heavy maintenance:} the more a script does its job well, the more
   you'll find yourself adding ``generic'' features and maintaining
   requests from other users.
\end{enumerate}
\end{frame}
\begin{frame}
\frametitle{What is GC3Libs?}
\label{sec-3}


  GC3Libs is a \href{http://www.python.org/}{Python} library to drive application execution on Grids
  and \href{http://www.oracle.com/us/products/tools/oracle-grid-engine-075549.html}{SGE} clusters.

  GC3Libs provides ways to customize execution control based on
  application type, and compose applications to form complex execution
  patterns.

  GC3Libs is part of a larger pack of tools called \href{http://gc3pie.googlecode.com/}{GC3Pie}.
  
\end{frame}
\begin{frame}
\frametitle{What is GC3Pie, then?}
\label{sec-4}

  \href{http://gc3pie.googlecode.com/}{GC3Pie} consists of:

\begin{itemize}
\item \emph{GC3Libs:} Python library, aimed at programmers to drive application
    execution on Grids and clusters.
\item \emph{GC3Utils:} simple command-line interface to the core GC3Libs
    functionality: submit/monitor/kill a job, retrieve output, etc.
\item \emph{GC3Apps:} Driver scripts developed for specific groups, but that
    may be of independent general interest.  (E.g., running the
    \href{http://www.rosettacommons.org/manuals/archive/rosetta3.1_user_guide/app_dock.html}{Rosetta Docking} application on a large set of inputs.)
\end{itemize}
\end{frame}
\begin{frame}
\frametitle{How is GC3Libs different? (I)}
\label{sec-5}

  GC3Libs runs specific \textbf{applications}, not generic jobs.

  That is, GC3Libs exposes \texttt{Application} classes whose programming
  interface is adapted to the specific task/computation a scientific
  application performs.

  GC3Libs supports a few applications in the main library.  (Our goal
  is to support more and more.)

  You can add your own applications.  You \emph{have to} add you own
  applications. 
\end{frame}
\begin{frame}
\frametitle{How is GC3Libs different? (II)}
\label{sec-6}

  GC3Libs can run applications in parallel, or sequentially, or any
  combination of the two, and do arbitrary processing of data in the
  middle.

  Think of \href{http://en.wikipedia.org/wiki/Scientific_workflow_system}{workflows}, except you can write them in the Python
  programming language.

  Which means, you can create them dynamically at runtime, adapting
  the schema to your problem.
\end{frame}
\begin{frame}
\frametitle{How could this solve any issues?}
\label{sec-7}

  \textbf{Portability:} GC3Libs aims at providing an abstraction over Grid
  and cluster resources: one single script is be able to run 
  on different computational sites.

  \textbf{Code reuse:} The application model, coupled with an object-oriented
  design, encourages writing more generic code that can be intergrated
  into the library.  (We hope for community-contributed code in the
  event.)

  \textbf{Heavy maintenance:} Generic features are part of the library. Focus
   on what makes your code special.
\end{frame}
\begin{frame}
\frametitle{GC3Libs application model}
\label{sec-8}

  An application is a subclass of the \texttt{gc3libs.Application} class.

  Generic \texttt{Application} class patterned after \href{http://www.nordugrid.org/documents/xrsl.pdf}{ARC's xRSL} model.

  At a minimum: provide application-specific command-line invocation.

  Advanced users can customize pre- and post-processing, react on
  state transitions, set computational requirements based on input
  files, influence scheduling.  (This is standard OOP: subclass and
  override a method.)
\end{frame}
\begin{frame}[fragile]
\frametitle{An example from the library}
\label{sec-9}

\begin{verbatim}
app = RosettaDockingApplication(
   100, # number of decoys to compute
   '1brs.pdb', # input file
   flags_file='flags.txt', # optional
)
\end{verbatim}

  The \texttt{RosettaDockingApplication} class knows how to invoke Rosetta's
  \texttt{docking\_protocol} program to compute N decoys of a given input file.
\end{frame}
\begin{frame}[fragile]
\frametitle{A basic example (I)}
\label{sec-10}

\begin{verbatim}
class SquareApplication(Application):
  """Compute the square of an integer, remotely."""
  def __init__(self, x):
    Application.__init__(
      self,
      executable = '/usr/bin/expr',
      arguments = [x, '*', x],
      inputs = [ ],
      outputs = [ ],
      stdout = "stdout.txt",
    )
\end{verbatim}
  This runs \texttt{expr x * x} and saves its output into \texttt{stdout.txt}
\end{frame}
\begin{frame}[fragile]
\frametitle{A basic example (II)}
\label{sec-11}

  When the remote computation is done, the \texttt{postprocess} method is
  called with the path to the output.
\begin{verbatim}
def postprocess(self, output_dir):
  output_file = open(output_dir + self.stdout)
  output_value = output_file.read()
  self.result = int(output_value)
\end{verbatim}
  The above code sets \texttt{result} to the integer value computed by
  running \texttt{expr x * x}.
\end{frame}
\begin{frame}
\frametitle{A simple high-throughput script structure\ldots{}}
\label{sec-12}

\begin{enumerate}
\item Get access to the Grid (e.g., authentication step)
\item Prepare files for submission
\item Submit jobs
\item Monitor job status (loop)
\item Retrieve results
\item Postprocess and display
\end{enumerate}
\end{frame}
\begin{frame}
\frametitle{Core operations}
\label{sec-13}

  Core operations: submit, update state, retrieve (a
  snapshot of) output, cancel job.

  Core operations are \textbf{synchronous}.
  
  Operations are always performed by a \texttt{Core} object.
  \texttt{Core} implements an overlay Grid on the resources 
  specified in the configuration file.
\end{frame}
\begin{frame}[fragile]
\frametitle{Core operations: verb/object interface}
\label{sec-14}

  Get an instance of \texttt{Core}:
\begin{verbatim}
g = Core(read_config_file(path))
\end{verbatim}

  Then you can operate on \texttt{Application} instances:
\begin{itemize}
\item submit: \texttt{g.submit(app)}
\item monitor: \texttt{g.update\_state(app)}
\item fetch output: \texttt{g.fetch\_output(app, dir)} (starts working as soon as
    application is RUNNING)
\item cancel job: \texttt{g.kill(app)}
\item free remote resources: \texttt{g.free(app)}
\end{itemize}
\end{frame}
\begin{frame}
\frametitle{A simple high-throughput script, GC3Libs version}
\label{sec-15}

\begin{enumerate}
\item \emph{Create a gc3libs.Core instance}
\item \emph{Create instance(s) of the application class}
\item Submit applications
\item Monitor application status (loop)
\item Retrieve results
\item Postprocess and display
\end{enumerate}
\end{frame}
\begin{frame}
\frametitle{What if\ldots{}?}
\label{sec-16}

  Looping is fine with a small number of jobs.

  What if I want to run 10'000 jobs in a session? Do I have to
  loop/wait until all of them are finished?

  What if my box crashes in the middle of the loop?  Do I lose all
  running jobs? 

  What if the proxy expires just in the middle of the loop?
\end{frame}
\begin{frame}
\frametitle{How do I manage authentication with GC3Libs?}
\label{sec-17}

  You don't.

  GC3Libs will check that there is always a valid proxy and
  certificate when attempting Grid operations, and if necessary, renew
  it.  

  GC3Libs provide a specific authentication module, that abstracts on
  the various authentication models.  It can be used to ease/automate
  authentication steps when accessing the Grid.
\end{frame}
\begin{frame}
\frametitle{Persisting jobs}
\label{sec-18}

  GC3Libs provides a simple persistence framework: 
\begin{itemize}
\item save a live \texttt{Application} to disk, return ``persistent ID''
\item load a saved application given its ``persistent ID''
\item delete a saved application
\item list IDs of saved applications (very simplistic! \textbf{your input     needed:} what kind of query/select operations should we support?)
\end{itemize}

  Filesystem-based storage (1 job, 1 file).  But interface is generic,
  could use SQL, \href{http://www.mongodb.org}{MongoDB}, etc.

  Implemented on top of Python's \texttt{pickle} module: it can persist any
  kind of object, not just jobs.
\end{frame}
\begin{frame}
\frametitle{Asynchronous operations}
\label{sec-19}

  The \texttt{Engine} class provides all core operations, 
  with a non-blocking interface.

  Calling core methods on an \texttt{Engine} instance returns immediately to
  the caller; operations are actually executed when you call the
  \texttt{Engine.progress()} method.

  Which you can do in a separate thread, thus achieving
  asynchronous operation.
\end{frame}
\begin{frame}
\frametitle{The \texttt{Engine} class}
\label{sec-20}

  Same programmatic interface as the \texttt{Core} class:
  you can use an \texttt{Engine} instance every time a \texttt{Core} is needed.

  The \texttt{progress()} method will advance jobs through their lifecycle; 
  use state-transition methods to take application-specific actions.
  (E.g., post-process output data.)
  
  An engine can automatically persist the jobs, if you so wish.
  (Just pass it a \texttt{Store} instance at construction time.)
\end{frame}
\begin{frame}
\frametitle{A high-throughput script with GC3Libs, revisited}
\label{sec-21}

\begin{enumerate}
\item \emph{Create a gc3libs.core.Core instance}
\item \emph{Create a gc3libs.persistence.FilesystemStore instance}
\item \emph{Create a gc3libs.core.Engine instance}
\item \emph{Load saved jobs into it}
\item Create \emph{new} instance(s) of the application class
\item \emph{Let engine manage jobs until all are done}
\item \st{Retrieve results} (the \texttt{Engine} does it)
\item Postprocess and display
\end{enumerate}

  Tasks 1.-4., and 6.-7. are automatically done by the
  \texttt{SessionBasedScript} class.
\end{frame}
\begin{frame}[fragile]
\frametitle{A high-throughput script with GC3Libs, \emph{the code}}
\label{sec-22}

\begin{verbatim}
core = Core(read_config_file)
store = FilesystemStore(directory)
apps = [ store.load(jobid) 
         for jobid in file(session, 'r') ]
for arg in new_args:
  apps += Application(arg, ...)
engine = Engine(core, apps, store)
engine.wait() # call progress() until done
total_global_postprocess()
\end{verbatim}

  This is the \textbf{actual} structure of the GRosetta/GGamess/GRunDB
  scripts!   (But you would just subclass \texttt{SessionBasedScript}.)
\end{frame}
\begin{frame}
\frametitle{Job dependency management}
\label{sec-23}

  GC3Libs provides \href{http://en.wikipedia.org/wiki/Directed_acyclic_graph}{Directed Acyclic Graph} support.

  DAGs are created programmatically from Python code.

  Which means, no graphical editor.  But also means you can create
  workflows on-the-fly as your computation proceeds.
\end{frame}
\begin{frame}
\frametitle{The \emph{3n+1} conjecture, a fictitious use case}
\label{sec-24}

  Define a function \emph{f(n)}, for \emph{n} positive integer:
\begin{itemize}
\item if \emph{n} is even, then \emph{f(n)=n} / \emph{2},
\item if \emph{n} is odd, then \emph{f(n)=3n+1},
\end{itemize}
  For every positive integer \emph{n}, form the sequence \emph{S(n)}:
  \emph{n} → \emph{f(n)} → \emph{f(f(n))} → \emph{f(f(f(n)))} → \ldots{}

  For example:
\begin{itemize}
\item \emph{S(1)} = 1 → 4 → 2 → 1,
\item \emph{S(2)} = 2 → 1,
\item \emph{S(3)} = 3 → 10 → 5 → 16 → 4 → 2 → 1,
\end{itemize}
\end{frame}
\begin{frame}
\frametitle{The \emph{3n+1} conjecture (II)}
\label{sec-25}

  \textbf{Conjecture:} For every positive integer \emph{n}, the sequence \emph{S(n)}
  eventually hits \emph{1}.

  Imagine you want to check the \emph{3n+1} conjecture up to some very
  large number \emph{N}.
\begin{itemize}
\item For each \emph{n = 1, \ldots{}, N}, run a task \emph{H(n)} that computes
    \emph{S(n)} until it hits \emph{1}.
\item Each task \emph{H(n)} is a sequence of computational jobs \emph{J(n,k)},
    where \emph{J(n,k+1)} applies function \emph{f} to the result of \emph{J(n,k)}.
\end{itemize}

  The \emph{J(n,k)} must be run \textbf{sequentially} (over \emph{k}).

  The \emph{H(n)} can be run in \textbf{parallel}.
\end{frame}
\begin{frame}[fragile]
\frametitle{The \emph{3n+1} conjecture (III)}
\label{sec-26}

  Let's define the simple application \emph{J(n,k)} that computes \emph{f}:
\begin{verbatim}
class HotpoApplication(Application):
  def __init__(self, n):
    Application.__init__(
      self,
      executable = '/usr/bin/expr',
      arguments = (
          # run `expr n / 2` if n is even
          [n, '/', n] if n % 2 == 0
          # run `expr 1 + 3 * n` if n is odd
          else [1, '+', 3, '*', n]),
      stdout = "stdout.txt",
    )
\end{verbatim}
  
\end{frame}
\begin{frame}
\frametitle{Composition of tasks (I)}
\label{sec-27}

  The unit of job composition is called a \texttt{Task} in GC3Libs.

  An \texttt{Application} is the primary instance of a \texttt{Task}.

  However, a single task can be composed of many applications.
  A task is a composite object: tasks can be composed of other tasks.

  Workflows are built by composing tasks in different ways.
  A ``workflow'' is a task, too.
\end{frame}
\begin{frame}
\frametitle{Composition of tasks (II)}
\label{sec-28}

  The \texttt{SequentialTask} class takes a list of jobs and executes them
  one after the other. Subclass and override the \texttt{next()} method to
  determine early exit conditions, or to modify the list dynamically.

  The \texttt{ParallelTask} class takes a list of jobs and executes all of
  them in parallel.  It's done when all jobs are done: there's an
  implicit synchronization barrier at the end.
  
\end{frame}
\begin{frame}
\frametitle{Composition of tasks (III)}
\label{sec-29}

  \texttt{Application}, \texttt{SequentialTask} and \texttt{ParallelTask} are all
  subclasses of the same \texttt{Task} interface.

  So, you can create sequential collections of parallel jobs, parallel
  collections of sequential collections, etc.

  Plus, collections can be mutated at run-time.

  An \texttt{Engine} really manages a list of tasks, so we are really
  scripting workflows here.
\end{frame}
\begin{frame}[fragile]
\frametitle{The \emph{3n+1} conjecture (IV)}
\label{sec-30}

  Now string together the \emph{J(n,k)} to compute a single sequence
  \emph{S(n)}:
\begin{verbatim}
class HotpoSequence(SequentialTask):
  def __init__(self, n):
    # compute first iteration of /f/
    self.tasks = [ MyApplication(n) ]
    SequentialTask.__init__(self, self.tasks)
  def next(self, k):
    last_computed_value = self.tasks[k].result
    if last_computed_value == 1:
      return TERMINATED
    else:
      self.tasks.append(MyApplication(last_computed_value)
       return RUNNING
\end{verbatim}
\end{frame}
\begin{frame}[fragile]
\frametitle{The \emph{3n+1} conjecture (V)}
\label{sec-31}

  Parallel tasks are independent by definition, so it's even easier to
  create a collection:
\begin{verbatim}
tasks = ParallelTaskCollection([ 
            HotpoSequence(n) for n in range(1, N) ])
\end{verbatim}
  
  We can run such a collection like any other \texttt{Application}.

  Have fun proving the \emph{3n+1} conjecture! ;-)
\end{frame}
\begin{frame}
\frametitle{A real use case: dynamic programming / value function iteration (I)}
\label{sec-32}

  Let a function \emph{U(t, a, b, c)} be given.  Assume that computing
  \emph{U} is computationally intensive.

  We want to compute a function \emph{f(x,t)} such that:
\begin{itemize}
\item \emph{f(x,0)} has a given value (boundary condition)
\item \emph{f(x,t+1) = U(t, f(x-1,t), f(x,t), f(x+1,t))}
\end{itemize}

  Can this be computed with a GC3Pie task collection?
\end{frame}
\begin{frame}
\frametitle{A real use case: dynamic programming / value function iteration (II)}
\label{sec-33}

  Computing \emph{f(x,t+1)} depends on the computed values of \emph{f(x±1,t)}.

  But computing \emph{f(x,t)} is independent of \emph{f(x',t)}.

  So we have a \textbf{sequence} of \textbf{parallel} jobs:
\begin{itemize}
\item a parallel collection \emph{P(t)} computes \emph{f(x,t)} for every \emph{x}
\item a sequential collection \emph{S} computes \emph{P(1)}, then \emph{P(2)}, etc\ldots{}
\end{itemize}
\end{frame}
\begin{frame}
\frametitle{Again on issues (I)}
\label{sec-34}

  \textbf{Portability:} different computational resources have different
  conceptual/programming models.  Taking the minimal common set of
  features is often not enough. 

  Solving this might need some radical programming change.
  Until that is sorted out, we have a constantly changing API, and
  even that might not support the features you need.
\end{frame}
\begin{frame}
\frametitle{Again on issues (II)}
\label{sec-35}

  \textbf{Code reuse:} At the end of the day, this very much depends on the
  user: so far users have been building sh/perl scripts to glue
  together Python scripts provided by GC3Libs.

  Is this ``just'' a language/habit issue?
\end{frame}
\begin{frame}
\frametitle{Again on issues (III)}
\label{sec-36}

  \textbf{Heavy maintenance:} GC3Libs provide classes for frequently-used
   patterns in high-throughput application scripting and programming.

   So, you get standard/generic features for free, and can concentrate
   on implementing the part that is specific to \emph{your} application.
\end{frame}
\begin{frame}
\frametitle{The usual disclaimer of warranty}
\label{sec-37}


  GC3Libs in active development.  Long list of features requests,
  planned improvements, and bug reports.

  Small team: work is being prioritized according to requests from
  users, but sometimes it takes quite some time from request to
  implementation.
  
\end{frame}
\begin{frame}
\frametitle{Any questions?}
\label{sec-38}

\begin{center}
GC3Pie home page: \href{http://gc3pie.googlecode.com}{http://gc3pie.googlecode.com}

Source code: \texttt{svn co http://gc3pie.googlecode.com/svn}

Mailing list: \href{mailto:gc3pie@googlegroups.com}{gc3pie@googlegroups.com}

\textbf{Thank you!}
\end{center}
\end{frame}
\begin{frame}
\frametitle{Technical details}
\label{sec-39}

  The following slides discuss a few technical details that complement
  the introduction.

  (But they are actually only relevant if you are trying to do some
  GC3Libs programming.)
\end{frame}
\begin{frame}
\frametitle{Application lifecycle}
\label{sec-40}

  GC3Libs \texttt{Application} objects mimic POSIX processes life-cycle.
  There's a \emph{single TERMINATED state}, whatever the job outcome.
  
  As with POSIX processes, you have to inspect the exit code and
  signals to determine the cause of ``job death''.
\begin{itemize}
\item If \texttt{os.WIFSIGNALED(app) = False} then job run to completion:
    check exit code!
\item If \texttt{os.WIFSIGNALED(app) = True} then some error occurred before
    end of application code.
\end{itemize}
  
  Grid- and batch-system errors are encoded as ``pseudo-signals''.
  E.g., if \texttt{os.WTERMSIG(app) = 124} then job was killed by remote
  batch system.
  
\end{frame}
\begin{frame}[fragile]
\frametitle{Core operations: self-action interface}
\label{sec-41}

  Get an instance of core, then ``attach'' an application to it:
\begin{verbatim}
g = Core(read_config_file())
app.attach(g)
\end{verbatim}

  The application can now operate on itself:
\begin{itemize}
\item submit: \texttt{app.submit()}
\item monitor: \texttt{app.update\_state()}
\item etc.
\end{itemize}

  Combined with state-transition methods, this gives a way to embed
  job control logic in the \texttt{Application} object.

  Think of automatic resubmission if certain conditions are met.
\end{frame}

\end{document}