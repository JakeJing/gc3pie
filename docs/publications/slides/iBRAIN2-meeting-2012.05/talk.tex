% Created 2011-03-14 lun 11:25
\documentclass[presentation]{beamer}
\usepackage[utf8]{inputenc}
\usepackage[T1]{fontenc}
\usepackage{fixltx2e}
\usepackage{graphicx}
\usepackage{longtable}
\usepackage{float}
\usepackage{wrapfig}
\usepackage{soul}
\usepackage{textcomp}
\usepackage{marvosym}
\usepackage{wasysym}
\usepackage{latexsym}
\usepackage{amssymb}
\usepackage{hyperref}
\tolerance=1000
\providecommand{\alert}[1]{\textbf{#1}}
\usepackage{pslatex}\usetheme{Madrid}\usecolortheme{default}\setlength{\parskip}{1em}
\begin{document}


\title{GC3Pie}
\author{Riccardo Murri, GC3, University of Zurich}
\date{2012-05-26}
\maketitle


\begin{frame}
  \frametitle{What is GC3Pie?}
  \label{sec-4}

  GC3Pie consists of:

  \begin{itemize}
  \item GC3Libs: Python library, aimed at programmers to drive application
    execution on Grids and clusters.
  \item GC3Utils: simple command-line interface to the core GC3Libs
    functionality: submit/monitor/kill a job, retrieve output, etc.
  \item GRosetta/GGamess/GRunDB/etc.: Driver scripts developed for
    specific groups, but that may be of independent general interest.
  \end{itemize}
\end{frame}


\begin{frame}
  \frametitle{GC3Libs application model}
  \label{sec-7}

  An application is a subclass of the \texttt{gc3libs.Application} class.

  Generic \texttt{Application} class patterned after \href{http://www.nordugrid.org/documents/xrsl.pdf}{ARC's xRSL} model.

  At a minimum: provide applications-specific command-line invocation.

  Advanced users can customize pre- and post-processing, react on
  state transitions, set computational requirements based on input
  files, influence scheduling.
\end{frame}

\begin{frame}[fragile]
  \frametitle{A stupid example}
  \label{sec-8}

\begin{verbatim}
class SquareApplication(Application):
  """Compute the square of an integer, remotely."""
  def __init__(self, x):
    Application.__init__(
      self,
      executable = '/usr/bin/expr',
      arguments = [x, '*', x],
      inputs = [ ],
      outputs = [ ],
      stdout = "stdout.txt",
    )
\end{verbatim}
\end{frame}

\begin{frame}[fragile]
  \frametitle{A less stupid example}
  \label{sec-9}

\begin{verbatim}
app = RosettaDockingApplication(
   100, # number of decoys to compute
   '1brs.pdb', # input file
   flags_file='flags.txt', # optional
)
\end{verbatim}

  The \texttt{RosettaDockingApplication} class knows how to invoke Rosetta's
  \texttt{docking\_protocol} program to compute N decoys of a given input file.
\end{frame}

\begin{frame}[fragile]
  \frametitle{A not-so-simple GC3Libs script (code)}
  \label{sec-20}

\begin{verbatim}
core = Core(read_config_file)
store = FilesystemStore(directory)
apps = [ store.load(jobid) 
         for jobid in file(session, 'r') ]
for arg in new_args:
  apps += Application(arg, ...)
engine = Engine(core, apps, store)
engine.wait() # call progress() until done
total_global_postprocess()
\end{verbatim}

  This is the \textbf{actual} structure of the GRosetta/GGamess/GRunDB
  scripts!   (But you would just subclass \texttt{SessionBasedScript}.)
\end{frame}

\begin{frame}
  \frametitle{Job dependency management}
  \label{sec-22}

  GC3Libs provides \href{http://en.wikipedia.org/wiki/Directed_acyclic_graph}{Directed Acyclic Graph} support.

  DAGs are created programmatically from Python code.

  Which means, no graphical editor.  But also means you can create
  workflows on-the-fly as your computation proceeds.
\end{frame}

\begin{frame}
  \frametitle{Composition of jobs (I)}
  \label{sec-23}

  The unit of job composition is called a \texttt{Task} in GC3Libs.

  An \texttt{Application} is the primary instance of a \texttt{Task}.

  However, a single task can be composed of many applications.
  A task is a composite object: tasks can be composed of other tasks.

  Workflows are built by composing tasks in different ways.
  A ``workflow'' is a task, too.
\end{frame}

\begin{frame}
  \frametitle{Composition of jobs (II)}
  \label{sec-24}

  The \texttt{SequentialTask} class takes a list of jobs and executes them
  one after the other. Subclass and override the \texttt{next()} method to
  determine early exit conditions, or to modify the list dynamically.

  The \texttt{ParallelTask} class takes a list of jobs and executes all of
  them in parallel.  It's done when all jobs are done: there's an
  implicit synchronization barrier at the end.
  
\end{frame}

\begin{frame}
  \frametitle{Composition of jobs (III)}
  \label{sec-25}

  \texttt{Application}, \texttt{SequentialTask} and \texttt{ParallelTask} are all
  subclasses of the same \texttt{Task} interface.

  So, you can create sequential collections of parallel jobs, parallel
  collections of sequential collections, etc.

  Any DAG can be expressed as composition of these classes.
  And collections can be mutated at run-time.

  An \texttt{Engine} really manages a list of tasks, so we are really
  scripting workflows here.
\end{frame}

\begin{frame}
\begin{center}
  GC3Pie home page: \href{http://gc3pie.googlecode.com}{http://gc3pie.googlecode.com}

  Source code: \texttt{svn co http://gc3pie.googlecode.com/svn}

  Mailing list: gc3pie@googlegroups.com

  Thank you!
\end{center}
\end{frame}


\end{document}