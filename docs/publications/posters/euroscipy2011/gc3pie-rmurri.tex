\documentclass[english,9pt]{beamer}
%\usepackage[T1]{fontenc}
\usepackage[utf8x]{inputenc}
\usepackage{babel}
%\usepackage{enumerate}

%% style
\usetheme{uzhneu-en-informal}
\usepackage{bookman}
\renewcommand{\H}[1]{\+{\Large\bfseries #1}\-}
\renewcommand\-{\vspace{0.5em}}
\newcommand\+{\vspace{1.5em}}

\begin{document}

% \title[GC3Pie]{Computational workflows with GC3Pie}
% \author{Riccardo Murri}
% \date{\today}

% \maketitle


\begin{frame}{%
    {Computational workflows with GC3Pie}
    \\
    {\url{http://gc3pie.googlecode.com/}}
  }%
\thispagestyle{empty}\rm

\H{What is GC3Pie?}

A Python library for driving large-scale distributed computations on
Grids and clusters.


\H{Why GC3Pie?}

\begin{columns}[t]
  \centering
  \begin{column}{.50\textwidth}
    {Workflows / task dependencies}

    {\em Task checkpoint / restart}
  \end{column}
  \begin{column}{.40\textwidth}
    {\em Code reuse}

    {Object orientation}
  \end{column}
\end{columns}

\H{The Poster}

A real-world example: optimization of an economic model running Ken Price's
``Differential Evolution'' algorithm.

\end{frame}
\end{document}

%%% Local Variables: 
%%% mode: latex
%%% TeX-master: t
%%% End: 
