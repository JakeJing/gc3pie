\documentclass[english,serif,mathserif,xcolor=pdftex,dvipsnames,table]{beamer}
\usetheme{gc3}

\usepackage[T1]{fontenc}
\usepackage[utf8]{inputenc}
\usepackage{babel}

\usepackage{gc3}

\title[Introduction]{%
  Introduction to \\ the GC3Pie training
}
\author[R. Murri, S3IT UZH]{%
  Riccardo Murri \texttt{<riccardo.murri@uzh.ch>}
  \\[1ex]
  \emph{S3IT: Services and Support for Science IT}
  \\[1ex]
  University of Zurich
}
\date{January~23--27, 2017}

\begin{document}

% title frame
\maketitle

\begin{frame}
  \begin{center}
    {\Huge Welcome!}
  \end{center}
\end{frame}


\begin{frame}
  \frametitle{What is GC3Pie?}
  GC3Pie is \ldots
  \begin{enumerate}
  \item \alert<2>{An \emph{opinionated} Python framework for defining and running computational workflows;}
  \item A \emph{rapid development toolkit} for running user applications on clusters and IaaS cloud resources;
  \item The worst name ever given to a middleware piece\ldots
  \end{enumerate}

  \+
  \uncover<2>{%
    As \emph{developers}, \alert<2>{you're mostly interested in this part.}
  }%
\end{frame}


\begin{frame}
  \frametitle{Who am I?}
  \begin{center}
    Systems administrator and programmer.
    \\ \+
    At UZH since 2010, first at GC3 then at S3IT.
    \\ \+
    Developer of GC3Pie since 2010.
  \end{center}
\end{frame}


\begin{frame}
  \begin{center}
    {\Huge and what about you?}
    \\ \+ \+
    \small
    Name $\blacksquare$ Affiliation $\blacksquare$ Interest in GC3Pie?
    $\blacksquare$~Programming languages?
    $\blacksquare$~Experience with command-line or terminal commands?
  \end{center}
\end{frame}


\begin{frame}
  \frametitle{Prerequisites}
  This course assumes:

  \begin{itemize}
  \item some experience with the Python programming language, and especially
    with its object-oriented constructs.
  \item familiarity with the Linux command-line shell.
  \end{itemize}

  \+
  Some prior exposure to GC3Pie (e.g., by running scripts or attending other
  courses) is not required but definitely \emph{helps.}
\end{frame}


\begin{frame}
  \frametitle{This is an \emph{interactive} course!}
  \begin{center}
    We'd like the training to be \\ as interactive and informal as
    possible.

    \+ \textbf{If you have a question, just ask -- don't wait.}

    \+
    After the course is over, \\ you're very welcome to keep asking questions:
    \\[1ex]
    on the mailing-list: \\
    \href{mailto:gc3pie@googlegroups.com}{gc3pie@googlegroups.com}
    \\[1ex]
    or through the forum web interface: \\
    {\small \url{https://groups.google.com/d/forum/gc3pie}}
  \end{center}
\end{frame}


\begin{frame}
  \frametitle{Where to find the course material}

  These slides and all the example files can be downloaded from the
  course web page at:
  {\small\url{http://tinyurl.com/gc3pie-workflows-tutorial}}

  \+
  A step-by-step guide to setting up your workstation for the course is at:
  {\small\url{http://tinyurl.com/sciencecloud-course-setup}}
\end{frame}


\begin{frame}
  \frametitle{Typographical conventions, I}

  The orange color is used for
  \href{http://tinyurl.com/gc3pie-july-2016}{clickable
    links}; this should make it easy to download sample files, etc.

  \+
  Other \hl{colors} and \HL{backgrounds} are used for highlighting
  text in slides.
\end{frame}


\begin{frame}[fragile]
  \frametitle{Typographical conventions, II}

    \begin{columns}[t]
    \begin{column}{0.5\textwidth}
\begin{lstlisting}
# This is how Python
# code looks like

def hello(name):
  print ("Hello, " + name)
\end{lstlisting}
    \end{column}
    \begin{column}{0.5\textwidth}
      \raggedleft Commentary text appears on the right.
    \end{column}
  \end{columns}
\end{frame}


\begin{frame}[fragile]
  \frametitle{Typographical conventions, III}

    \begin{columns}[t]
    \begin{column}{0.5\textwidth}
      \begin{sh}
$ echo hello
hello
      \end{sh}%$
\begin{python}
>>> print 'hello'
hello
\end{python}
    \end{column}
    \begin{column}{0.5\textwidth}
      \raggedleft
      Commands to type in the terminal shell are signalled by the `\texttt{\$}' prompt

      \+
      Commands to type in the Python shell are signalled by the `\texttt{>{}>{}>}' prompt
    \end{column}
  \end{columns}
\end{frame}

\begin{frame}
  \frametitle{GC3Pie documentation}

  The full GC3Pie documentation can be found at
  \url{http://gc3pie.readthedocs.io/}

  \+
  This tutorial assumes you're running GC3Pie version~2.5
  (i.e., currently the \emph{master} branch on GitHub).
\end{frame}


\end{document}

%%% Local Variables:
%%% mode: latex
%%% TeX-master: t
%%% End:
