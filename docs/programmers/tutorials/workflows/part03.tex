\documentclass[english,serif,mathserif,xcolor=pdftex,dvipsnames,table]{beamer}
\usetheme{gc3}

\usepackage[T1]{fontenc}
\usepackage[utf8]{inputenc}
\usepackage{babel}

\usepackage{gc3}

\title[Introduction]{%
  Useful debugging commands
}
\author[R. Murri, S3IT UZH]{%
  Riccardo Murri \texttt{<riccardo.murri@uzh.ch>}
  \\[1ex]
  \emph{S3IT: Services and Support for Science IT}
  \\[1ex]
  University of Zurich
}
\date{July~11--14, 2016}

\begin{document}

% title frame
\maketitle


\begin{frame}[fragile]
  \frametitle{List tasks in a session}

  The \texttt{gsession list} command prints the list of tasks in a
  session, together with their state:

\begin{stdout}
$ ^\HL{gsession list -r ex2b}^
+----------------+-----------------+------------+---------------------------------------+
| JobID          | Job name        | State      | Info                                  |
+----------------+-----------------+------------+---------------------------------------+
| GrayscaleApp.6 | GrayscaleApp-N1 | TERMINATED | TERMINATED at Sat Jul 9 22:13:58 2016 |
+----------------+-----------------+------------+---------------------------------------+
\end{stdout}%$
\end{frame}


\begin{frame}[fragile]
  \frametitle{Show log of actions in session}

  The \texttt{gession log} command prints out the sequence of actions
  and state changes for all tasks in a session:

\begin{stdout}
$ ^\HL{gsession log ex2b}^
Jul 09 22:13:53 GrayscaleApp.6: Submitting to 'localhost'
Jul 09 22:13:53 GrayscaleApp.6: SUBMITTED
Jul 09 22:13:53 GrayscaleApp.6: Submitted to 'localhost'
Jul 09 22:13:58 GrayscaleApp.6: TERMINATING
Jul 09 22:13:58 GrayscaleApp.6: Final output downloaded to 'grayscale.d'
Jul 09 22:13:58 GrayscaleApp.6: TERMINATED
\end{stdout}%$
\end{frame}


\begin{frame}[fragile]
  \frametitle{Dump contents of a task}

  The \texttt{ginfo} command allows you to dump the contents of a
  specific task, or all tasks in a session:

\begin{stdout}[basicstyle=\tiny\ttfamily]
$ ^\HL{ginfo -v -s ex2b}^
GrayscaleApp.6
  arguments: convert, lena.jpg, -colorspace, gray, gray-lena.jpg
  ^\em [\ldots]^
  execution:
    ^\em [\ldots]^
    history:
      - Submitting to 'localhost' at Sun Jul 10 00:27:02 2016
      ^\em [\ldots]^
    lrms_execdir: /home/gc3pie/docs/programmers/tutorials/workflows/gc3libs.SMGNxr
    lrms_jobid: 23183
    resource_name: localhost
    ^\em [\ldots]^
  inputs:
    file:///home/gc3pie/docs/programmers/tutorials/workflows/lena.jpg: lena.jpg
  ^\em [\ldots]^
  output_dir: grayscale.d
  outputs:
    gray-lena.jpg: file, , gray-lena.jpg, None, None, , None, None
    stderr.txt: file, , stderr.txt, None, None, , None, None
    stdout.txt: file, , stdout.txt, None, None, , None, None
  ^\em [\ldots]^
  stderr: stderr.txt
  stdin: None
  stdout: stdout.txt
  ^\em [\ldots]^
\end{stdout}%$
\end{frame}


\begin{frame}[fragile]
  \frametitle{Dump contents of a task, II}

  Note that \textbf{without the \texttt{-v}} option, \texttt{ginfo}
  limits its output to the \texttt{.execution} attribute of a
  \texttt{Task}/\texttt{Application} object:

\begin{stdout}[basicstyle=\tiny\ttfamily]
$ ^\HL{ginfo -s ex2b}^
GrayscaleApp.6
  ^\em [\ldots]^
  history:
    - Submitting to 'localhost' at Sun Jul 10 00:27:02 2016
    ^\em [\ldots]^
  lrms_execdir: /home/gc3pie/docs/programmers/tutorials/workflows/gc3libs.SMGNxr
  lrms_jobid: 23183
  resource_name: localhost
  ^\em [\ldots]^
\end{stdout}%$
\end{frame}


\begin{frame}[fragile]
  \frametitle{Abort all tasks in a session}

  The \texttt{gsession abort} command kills all tasks in a session.
\begin{stdout}
$ ^\HL{gsession abort ex2b}^
\end{stdout}%$

  \+
  It produces normally no output; use the \texttt{-v} option to see a
  log of actions taken.

  \+ \small \textbf{Note:} it is important that sessions are
  terminated!  Otherwise, GC3Pie will consider part of the resources
  as still allocated to a task.  This is especially evident when
  running on a single computer: after launching a few tasks, GC3Pie
  will stop and refuse to run anything.
\end{frame}

\end{document}

%%% Local Variables:
%%% mode: latex
%%% TeX-master: t
%%% End:
