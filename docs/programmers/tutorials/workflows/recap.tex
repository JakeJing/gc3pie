\documentclass[english,serif,mathserif,xcolor=pdftex,dvipsnames,table]{beamer}
\usetheme{gc3}

\usepackage[T1]{fontenc}
\usepackage[utf8]{inputenc}
\usepackage{babel}

\usepackage{gc3}

\title[Recap]{%
  Take-home messages \\ from each training day
}
\author[R. Murri, S3IT UZH]{%
  Riccardo Murri \texttt{<riccardo.murri@uzh.ch>}
  \\[1ex]
  \emph{S3IT: Services and Support for Science IT}
  \\[1ex]
  University of Zurich
}
\date{January~23--27, 2017}

\begin{document}

% title frame
\maketitle


\begin{frame}[fragile]
  \frametitle{Recap of Day 1}

  \begin{enumerate}
  \item \texttt{SessionBasedScript} is the ``engine'' to run GC3Pie workflows and task collections.

  \+\item A script must be fed a list of tasks to execute by the \lstinline|new_tasks()| method.

  \+\item GC3Pie code runs unmodified on different computational resources.
  \end{enumerate}
\end{frame}


\begin{frame}[fragile]
  \frametitle{Recap of Day 2}

  \begin{enumerate}
  \item Utility commands \texttt{gsession} and \texttt{ginfo} help
    inspect a running session.  To debug problems, start with running
    the session-based script with maximum verbosity:
    \texttt{-{}-verbose} \texttt{-{}-verbose} \texttt{-{}-verbose}

    \+\item Command-line argument and option processing is defined through
    methods \lstinline|setup_args| and \lstinline|setup_options| of the
    \texttt{SessionBasedScript}

    \+\item Use \lstinline|requested_cores|,
    \lstinline|requested_memory|, \lstinline|requested_walltime| to
    specify environmental requirements of each task.
  \end{enumerate}
\end{frame}


\begin{frame}
  \frametitle{Recap of Day 3}

  \begin{enumerate}
  \item Post-processing can be done:
    \begin{itemize}
    \item in the \texttt{terminated()} method of the \texttt{Application} (or \texttt{Task}) class, or
    \item globally in the \lstinline|after_main_loop()| method of the \texttt{SessionBasedScript} class.
    \end{itemize}
  \item Exit code and termination status inspection.
  \end{enumerate}
\end{frame}


\begin{frame}
  \frametitle{Recap of Day 4}

  \begin{enumerate}
  \item Applications can be composed into workflows using ``task collections'':
    \begin{itemize}
    \item \texttt{SequentialTaskCollection} is for a series of tasks that should
      be executed \emph{in the order given}
    \item \texttt{ParallelTaskCollection} is for tasks with no inter-dependency
      (so, all potentially running at the same time)
    \end{itemize}

  \item task collections are tasks themselves, so collections can be nested to
    create any kind of directed graph

  \item \texttt{StagedTaskCollection} is for creating processing pipelines where
    all the steps are \emph{known before runtime}
  \end{enumerate}
\end{frame}


\begin{frame}
  \frametitle{Recap of Day 5}

  \begin{itemize}
  \item \texttt{ParallelTaskCollection} is for sets of tasks that can all
    execute in parallel --- with no inter-dependency or communication.
  \item \texttt{DependentTaskCollection} is for tasks whose dependencies are
    known before runtime
  \item The \texttt{next()} method of \texttt{SequentialTaskCollection} can be
    used to create ``dynamic'' sequences that change while running.
  \end{itemize}

\end{frame}


\begin{frame}
  \frametitle{...and after the course?}
  \centering

  All VMs to be deleted tomorrow morning. \\
  So, \textbf{copy all data you want to save} today!

  \+ 
  Your account will be removed \\
  from the \texttt{training} project on Science Cloud. \\
  (Any other project membership stays.)

  \+
  For all things GC3Pie: visit our offices (Y11 F 66) \\
  or \textbf{send email to \texttt{help@s3it.uzh.ch}}.
\end{frame}


\begin{frame}{}
  \begin{center}
    \Large Please provide your feedback \\ about this course: \\
    \url{http://tinyurl.com/gc3pie-feedback-nov-2016}
  \end{center}
\end{frame}

\end{document}

%%% Local Variables:
%%% mode: latex
%%% TeX-master: t
%%% End:
