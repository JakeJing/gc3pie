\documentclass[english,serif,mathserif,xcolor=pdftex,dvipsnames,table]{beamer}
\usetheme{gc3}

\usepackage[T1]{fontenc}
\usepackage[utf8]{inputenc}
\usepackage{babel}


\usepackage{gc3}

\title[Requirements]{%
  Application requirements
}
\author[R. Murri, S3IT UZH]{%
  Riccardo Murri \texttt{<riccardo.murri@uzh.ch>}
  \\[1ex]
  \emph{S3IT: Services and Support for Science IT}
  \\[1ex]
  University of Zurich
}
\date{July~11--14, 2016}


\begin{document}

% title frame
\maketitle


\begin{frame}
  Applications need to allocate computing resources.
  For instance, request 4 processors for 8 hours.

  \+
  GC3Pie allows requesting:
  \begin{itemize}
  \item the number of processors that a job can use,
  \item the architecture (32-bit or 64-bit) of these processors,
  \item the guaranteed duration of a job,
  \item the amount of memory that a job can use (per processor).
  \end{itemize}

  \+
  More fine-grained matching is possible, but outside the scope of
  this introductory training.
\end{frame}


\begin{frame}[fragile]
  Resources are requested using additional constructor parameters for
  \texttt{Application} objects.

  \+
  The allowed parameters are:
  \lstinline|requested_cores|,
  \lstinline|requested_architecture|,
  \lstinline|requested_walltime|,
  \lstinline|requested_memory|.
\end{frame}


\begin{frame}[fragile]
  \frametitle{Running parallel jobs}

  You request allocation of a certain number of processors using the
  \lstinline|requested_cores| parameter: set it to the number of CPU
  cores that you want.

  \+
  For example, the following runs the command \texttt{mpixexec
    simulator} on 4 processors:
  \begin{python}
  class ZodsApplication(Application):
    # ...
    Application.__init__(self,
      arguments=['mpiexec', '-n', '4', 'simulator'],
      # ...
      ~\HL{requested\_cores=4}~)
  \end{python}

  \+
  {\small Note that GC3Pie only guarantees the availability of a certain
    number of processors; it is your application's responsibility to use
    them, e.g., by starting a command using MPI or any other parallel
    processing mechanism.}
\end{frame}


\begin{frame}[fragile]
  \frametitle{Requesting processor architecture}

  If you send the compiled executable along with your application, you
  need to select only resources that can run that binary file.

  \+
  The \lstinline|requested_architecture| parameter provides the
  choice between \lstinline|gc3libs.Run.Arch.X86_64| (for 64-bit
  Intel/AMD computers) and \lstinline|gc3libs.Run.Arch.X86_32| (for
  32-bit ones):
  \begin{python}
  from gc3libs import Run
  class CodemlApplication(Application):
    # ...
    Application.__init__(self,
      arguments=['./codeml.bin'],
      inputs = ['/usr/local/bin/codeml', ...]
      # ...
      ~\HL{requested\_architecture=Run.Arch.X86\_64}~)
  \end{python}
\end{frame}


\begin{frame}[fragile]
  \frametitle{Requesting running time}

  In order to ensure that your job is allotted enough time to run on
  the remote computing system, use the \lstinline|requested_walltime|
  parameter.

  \+
  \begin{python}
  ~\HL{\textbf{from} gc3libs.quantity}~ \
    ~\HL{\textbf{import} minutes, hours, days}~

  class CodemlApplication(Application):
    # ...
    Application.__init__(self,
      # ...
      ~\HL{requested\_walltime=8*hours}~)
  \end{python}

  \+
  You \textbf{must} use a \texttt{gc3libs.quantity} multiple for the
  \lstinline|requested_walltime| parameter; any other value will be
  rejected with an error.
\end{frame}

\begin{frame}[fragile]
  \frametitle{Units of time}
  The Python module \texttt{gc3libs.quantity} provides units for
  expressing time requirements in days, hours, minutes, seconds.

  \+
  Just multiply the unit by the amount you need:
  \begin{python}
    >>> an_hour = 1*hours
  \end{python}
  Or sum the amounts:
  \begin{python}
    >>> two_days = 1*days + 24*hours
  \end{python}

  \+
  GC3Pie will automatically perform the conversions:
  \begin{python}
    >>> two_hours = 2*hours
    >>> another_two_hours = 7200*seconds
    >>> two_hours == another_two_hours
    True
  \end{python}
\end{frame}


% \begin{frame}
%   \frametitle{CPU time vs wall-clock time}

%   ``CPU time'' is the total time spent by all CPUs in the system
%   actually executing code from our job.

%   \+
%   The ``wall-clock time'' (abbr. ``walltime'') is the time that
%   passes on a clock from the moment the system starts executing a job
%   until the end of that job.
% \end{frame}


\begin{frame}[fragile]
  \frametitle{Requesting memory}
  In order to secure a certain amount of memory for a job, use the
  \lstinline|requested_memory| parameter.

  \+
  Example:
\begin{python}
  ~\HL{\textbf{from} gc3libs.quantity \textbf{import} GB, MB, kB}~
  class CodemlApplication(Application):
    # ...
    Application.__init__(self,
      # ...
      ~\HL{requested\_memory=8*GB}~)
\end{python}

  \+
  Note that \lstinline|requested_memory| expresses the total
  memory used by the job!
\end{frame}

\begin{frame}[fragile]
  \frametitle{Units of memory}
  The Python module \texttt{gc3libs.quantity} provides units for
  expressing memory requirements in kilo-, Mega- and Giga-bytes.

  \+
  Just multiply the unit by the amount you need:
\begin{python}
  >>> a_gigabyte = 1*GB
  >>> two_megabytes = 2*MB
\end{python}

  \+
  GC3Pie will automatically perform the conversions:
  \begin{python}
    >>> two_gigabytes = 2*GB
    >>> another_two_gbs = 2000*MB
    >>> two_gigabytes == another_two_gbs
    True
  \end{python}
\end{frame}


\begin{frame}[fragile]
  \frametitle{All together now}

\begin{python}
from gc3libs.quantity import GB, MB, kB
from gc3libs.quantity import days, hours, minutes

class CodemlApplication(Application):
  # ...
  Application.__init__(self,
    # ...
    requested_cores=1,
    requested_memory=2*GB,
    requested_walltime=8*hours)
\end{python}

  \+ When several resource requirements are specified, GC3Pie tries to
  satisfy \emph{all} of them.  \textbf{If this is not possible, task
  submission fails and the task stays in state \emph{NEW}.}

\end{frame}


% \begin{frame}
%   \begin{exercise}[5.A]
%     Modify the \texttt{GrayscaleApp} in
%     \href{https://github.com/uzh/gc3pie/blob/master/docs/programmers/tutorials/workflows/solutions/ex2c.py}{ex2c.py}
%     so that it requests an impossibly high amount of memory. Re-run
%     the script and watch it fail: the application should not be
%     submitted and remain in state \emph{NEW}.

%     \+
%     Can you achieve the same result by using other requirement specifiers?
%   \end{exercise}
% \end{frame}


\end{document}

%%% Local Variables:
%%% mode: latex
%%% TeX-master: t
%%% End:
